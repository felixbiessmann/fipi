%
% File poltext2016.tex
%
% Contact: jan.snajder@fer.hr or dsirinic@gmail.com
%%
%% Based on the style files for ACL-2015, which were, in turn,
%% Based on the style files for ACL-2014, which were, in turn,
%% Based on the style files for ACL-2013, which were, in turn,
%% Based on the style files for ACL-2012, which were, in turn,
%% based on the style files for ACL-2011, which were, in turn, 
%% based on the style files for ACL-2010, which were, in turn, 
%% based on the style files for ACL-IJCNLP-2009, which were, in turn,
%% based on the style files for EACL-2009 and IJCNLP-2008...

%% Based on the style files for EACL 2006 by 
%%e.agirre@ehu.es or Sergi.Balari@uab.es
%% and that of ACL 08 by Joakim Nivre and Noah Smith

\documentclass[11pt]{article}
\usepackage{poltext2016}
\usepackage{times}
\usepackage{url}
\usepackage{latexsym}
\usepackage{booktabs}
\usepackage{graphicx} % more modern
%\usepackage{epsfig} % less modern
\usepackage{subfigure}

\usepackage{natbib}
\usepackage{amsmath,amsfonts,amscd,amssymb}
\usepackage{dsfont}
\renewcommand{\vec}[1]{\mathbf{#1}}
\usepackage{hyperref}
\DeclareMathOperator*{\argmax}{argmax}
\DeclareMathOperator*{\argmin}{argmin}
\DeclareMathOperator*{\Corr}{Corr}
\newcommand{\R}{\mathds{R}}
\usepackage{multicol}
\usepackage{multirow}
\usepackage{pbox}
\usepackage{color}

% As of 2011, we use the hyperref package to produce hyperlinks in the
% resulting PDF.  If this breaks your system, please commend out the
% following usepackage line and replace \usepackage{icml2015} with
% \usepackage[nohyperref]{icml2015} above.
\usepackage{hyperref}

% Packages hyperref and algorithmic misbehave sometimes.  We can fix
% this with the following command.
\newcommand{\theHalgorithm}{\arabic{algorithm}}


\newcommand{\pola}[1]{\textcolor{blue}{#1}}
\newcommand{\felix}[1]{\textcolor{green}{#1}}

%\setlength\titlebox{5cm}

% You can expand the titlebox if you need extra space
% to show all the authors. Please do not make the titlebox
% smaller than 5cm (the original size); we will check this
% in the camera-ready version and ask you to change it back.

\title{Predicting political party affiliation from text}

\author{Felix Biessmann\\
  {\tt felix.biessmann@gmail.com} \\ \And
  Sebastian Schelter\\
  Technische Universit\"at Berlin\\
  {\tt sebastian.schelter@tu-berlin.de}\\
  \And
 Pola Lehmann \\
  Wissenschaftszentrum Berlin \\
  {\tt pola.lehmann@wzb.eu} \\}

\date{}

\begin{document}
\maketitle


\begin{abstract}
Every day large amounts of text are produced in public discourse. Some of this text is produced by actors whose political colour is very obvious, for example in the case of party elites. But there are also many actors raising their voices in the discourse, who cannot be clearly attributed to a political party, while their statements might actually by biased towards a specific party. Identifying such biases is crucial for political research as well as media consumers, especially when analysing the influence of the media on political discourse and vice versa. In this study we investigate to what extent political party affiliation can be predicted from textual content. Results indicate that automatic classification of political affiliation is possible with above chance accuracy also across text domains. We propose methods to better interpret these results and find that features not related to political policies, such as speech sentiment, can be discriminative and thus exploited by text analysis models. We present an example of a web application that combines topic modelling with automatic political affiliation analyses of news articles based on the proposed model.
\end{abstract}

\section{Introduction}
\label{sec:intro}
%
Analysis and classifications of political text is and has been a very important tool to generate political science data (Benoit et al. 2015). Traditionally such classifications are done by experts, who read and label the text of interest.\footnote{See for example the Manifesto Project (Budge et al. 2001; Klingemann et al. 2006), the Comparative Agendas Project or Poltext (Pétry/Duval 2015).} This is, however, a very time consuming task and thus sets various limits to the possible amount of data that a few experts can analyze. The growing field of automated text analysis, that allows the analysis of much more text in less time, is therefore of great interest to political scientists. Additionally automated text analyses allow for a more objective and replicable analysis of political text then human coders could achieve (Benoit et al. 2009)

One area which produces large amounts of text which are of high interest for political scientists is modern media. Standard newspapers, but also new forms as twitter, facebook or blogs produce an ever growing amount of text influencing political debate. These texts can be of interest to political scientists in various analyses. The question we will focus on in this analysis is the potential political bias that can be detected in these sources. In many cases it is obvious which political bias an author has. In other cases some expertise is required to judge the political bias of a text. Keeping an unbiased view on what media report on requires to understand the political bias of texts. Different newspaper outlets are, for example, often ascribed as favoring different positions on the political left-right axis. Data from the European Media Systems Survey show that experts clearly associate different media outlets with specific parties (http://www.mediasystemsineurope.org). However such expert surveys on the political bias of newspapers can normally only give us a very general tendency of a newspaper, they do not distinguish between different journalists or different sections. Assistive technology can help in this context to try and obtain a more unbiased sample of information. It gives us the possibility to get a more nuanced view of the spectrum of political positions favored in important news outlets in a country. In this paper we test whether it is possible to automatically detect political party affiliation in text and if such a model can be used on different kinds of texts that it was trained on. 
In order to validate and explain the predictions of the models three strategies that allow for better interpretations of the models are proposed. First the model misclassifications are related to changes in party policies. Second univariate measures of correlation between text features and party affiliation allow to relate the predictions to the kind of information that political experts use for interpreting texts. Third sentiment analysis is used to investigate whether this aspect of language has discriminatory power.

In the following \autoref{sec:data} gives an overview of the data acquisition and preprocessing methods, \autoref{sec:model} presents the model, training and evaluation procedures; in \autoref{sec:results} the results are discussed and \autoref{sec:conclusion} concludes with some interpretations of the results and future research directions.

\section{Data Sets and Feature Extraction}\label{sec:data}
%
All experiments were run on publicly available data sets of German political texts and standard libraries for processing the text. The following sections describe the details of data acquisition and feature extraction.

\subsection{Data}
Annotated political text data was obtained from two sources: a) the plenary debates held in the German parliament ({\em Bundestag}) and b) all manifesto texts of parties winning seats in the election to the German parliament in the the last, 17th, legislative period. 

\paragraph{Parliament discussion data} Parliament texts are annotated with the respective party label, which we take here as a proxy for political bias. The protocols of plenary debates are available through the website of the German Bundestag \cite{bundestag}; an open source API was used to query the data in a cleaned and structured format \cite{bundestag-github}. In total 22784 speeches were extracted for the 17th legislative period\footnote{Uninterrupted parts were treated as separate speech.}. 

\paragraph{Party manifesto data}
The party manifesto text was taken from the Manifesto Corpus  \cite{manifesto}. The data released in this project mainly comprises the complete manifestos of all parties that have won seats at a national election. Each quasi-sentence\footnote{A quasi-sentence has the length of an argument. It is never longer than one sentence.} is annotated with one of 56 policy issue categories. Examples for the policy categories are {\em welfare state expansion, welfare state limitation, democracy, equality}; for a complete list and detailed explanations on how the annotators were instructed see \cite{leftright}. Each quasi-sentence has  two types of labels: the party affiliation and the manually assigned policy issue aimed at in each quasi-sentence. The policy labels were used to aggregate the data into the following topics: {\em External Relations, Freedom and Democracy, Political System, Economy, Welfare and Quality of Life, Fabric of Society, Social Groups}. 
The length of each annotated statement in the party manifestos is rather short. The median length is 95 characters or 12 words.\footnote{The longest statement is 522 characters (65 words) long, the 25\%/50\%/75\% percentiles are 63/95/135 characters or 8/12/17 words, respectively.} 

\subsection{Bag-of-Words Vectorization}\label{sec:bow-vectorization}
All text data was tokenised and transformed into bag-of-word (BOW) vectors as implemented in scikit-learn \cite{scikit-learn}. Several options for BOW vectorizations were tried, including term-frequency-inverse-document-frequency normalisation, n-gram patterns up to size $n=3$ and several cutoffs for discarding too frequent and too infrequent words. All of these hyperparameters were subjected to hyperparameter optimization as explained in \autoref{sec:crossvalidation}.

\section{Classification Model and Training}\label{sec:model}
Bag-of-words feature vectors were used to train a multinomial logistic regression model. Let $y\in\{1,2,\dots,K\}$ be the true  label, where $K$ is the total number of labels and $\vec{W}=[\vec{w}_1,\dots,\vec{w}_K]\in\R^{d\times K}$ is the concatenation of the weight vectors $\vec{w}_k$ associated with the $k$th party then
\begin{eqnarray}\label{eq:logreg_multiclass}
p(y=k|\vec{x},\vec{W}) = \frac{e^{z_k}}{\sum_{j=1}^K e^{z_j}}  \textrm{ with }  z_k=\vec{w}_k^{\top}\vec{x}.
\end{eqnarray}

\subsection{Optimisation of Model Parameters}\label{sec:crossvalidation}
The model pipeline contained a number of  hyperparameters that were optimised using cross-validation. The parliament speech data was split into training and validation (for hyperparameter optimization) in a 90\%/10\% ratio. 
%
\subsection{Sentiment analysis}\label{sec:sentiment_analysis_methods}
A publicly available key word list was used to extract sentiments \cite{remquahey2010}. A sentiment vector $\vec{s}\in\R^d$ was constructed from the sentiment polarity values in the sentiment dictionary. The sentiment index used for attributing positive or negative sentiment to a text was computed as the cosine similarity between BOW vectors and sentiment vectors.

\subsection{Interpreting bag-of-words models}\label{sec:correlations_methods}
Interpreting coefficients of linear models (independent of the regularizer used) implicitly assumes uncorrelated features; this assumption is violated by the text data used in this study. Thus direct interpretation of the model coefficients $\vec{W}$ is problematic, see also \cite{Zien2009, Haufe2013}. In order to allow for better interpretation of the predictions and to assess which features are discriminative correlation coefficients between each word and the party affiliation label were computed. The words corresponding to the top positive and negative correlations are shown in \autoref{sec:word_party_correlations}.

\section{Results}\label{sec:results}

The following section gives an overview of the results for all political bias prediction tasks. 
%Some interpretations of the results are highlighted and a web application of the models is presented at the end of the section. 
Due to space restrictions we only report results from the 17th legislative period. Results are similar for the 18th legislative period.

\subsection{Predicting political party affiliation}
The evaluation results for the political party affiliation prediction on in-domain data (held-out parliamentary speech text) for the 17th Bundestag are listed in \autoref{tab:results_in-domain}.

When predicting party affiliation on text data from the same domain that was used for training the model, average precision and recall values of above 0.6 are obtained. These results are comparable to those of \cite{Hirst2014} who report a classification accuracy of 0.61 on a five class problem predicting party affiliation in the European parliament.
For out-of domain data the models yield significantly lower precision and recall values between 0.3 and 0.4 \autoref{tab:results_out-of-domain}. This drop in out of domain prediction accuracy is in line with previous findings \cite{Yu2008}.
\paragraph{Length of texts influences prediction accuracy}
A main factor that made the prediction on the out-of-domain prediction task particularly difficult is the short length of the strings to be classified, see also \autoref{sec:data}. In order to investigate whether this low out-of-domain prediction performance was due to the domain difference (parliament speech vs manifesto data) or due to the short length of the data points, the manifesto data was aggregated into political topics. The topic level results are shown in \autoref{tab:results_topic} and demonstrate that when the texts to be classified are sufficiently long and the word count statistics are sufficiently dense the classification performance on out-of-domain data can - at least in some cases - achieve reliable precision and recall values close to 1.0. This increase is in line with previous findings on the influence of text length on political bias prediction accuracy \cite{Hirst2014}.


\begin{table}[t]
\caption{
\label{tab:results_in-domain}
{\bf In-domain classification performance} for data from the 17th legislative period on in-domain data. $N$ denotes number of data points in the evaluation set.
}
\begin{center}
\begin{tabular}{lcccc}
    &         precision    &recall &  f1-score  & N  \\
\hline \hline
       cducsu   &    0.62  &    0.81  &    0.70  &     706\\
        fdp    &   0.70   &   0.37  &    0.49    &   331\\
     gruene &      0.59  &    0.40   &   0.48   &    298\\
      linke    &   0.71   &   0.61  &    0.65    &   338\\
        spd   &    0.60   &   0.69  &    0.65   &    606\\
\hline
 total &      0.64   &   0.63   &   0.62    &  2279 
%
\end{tabular}
\end{center}
\end{table}

\begin{table}[t]
\caption{
\label{tab:results_out-of-domain}
{\bf Out-of-domain classification performance} (quasi-sentence level) of a classifier trained on speeches of the 17th legislative period of the Bundestag.
}

\begin{center}
\begin{tabular}{lcccc}
    &         prec.    &recall &  f1-score  & N  \\
\hline \hline
    cducsu    &   0.26   &   0.58   &   0.36    &   2030 \\
    fdp    &   0.38   &   0.28   &   0.33    &   2319 \\
     gruene   &    0.47    &  0.20   &   0.28    &  3747\\
      linke     &  0.30  &    0.47    &  0.37    &   1701\\
        spd     &  0.26  &    0.16   &   0.20    &   2278\\
\hline
total    &   0.35  &    0.31  &    0.30   &   12075\\
%
\end{tabular}
\end{center}

\end{table}

\begin{table}[t]
\caption{
\label{tab:results_topic}
{\bf Out-of-domain classification performance} (topic level) on manifesto data of a classifier trained on the 17th legislative period of the Bundestag. Compared to quasi-sentence level predictions (\autoref{tab:results_out-of-domain}) the predictions made on topic level are more reliable.}
\begin{center}
\begin{tabular}{lcccc}
    &         precision    &recall &  f1-score  & N  \\
    \hline
        \hline
cducsu     &  0.64  &    1.00  &    0.78    &     7\\
       fdp    &   1.00    &  1.00    &  1.00    &     7\\
    gruene  &     1.00  &    0.86  &    0.92    &     7\\
     linke    &   1.00   &   1.00     & 1.00    &     7\\
       spd   &    0.80   &   0.50    &  0.62     &    8\\
    \hline
total  &     0.88   &   0.86   &   0.86  &      36\\
\end{tabular}
\end{center}

\end{table}

\paragraph{Policy change}
Another explanation for misclassification could lie in the fact that parties change their policy positions. In order to investigate this confusion matrices were extracted for the predictions on the out-of-domain evaluation data for topic level predictions (see \autoref{tab:confusion_topic}). On the topic level results for the 17th legislative period are pretty good except for the SPD, which the classifier often predicts as CDU/CSU. One explanation for this misclassification could be policy changes. \\

% Confusion matrix topic level
\begin{table}[t]\label{tab:conf_mat_four_class}
\caption{\label{tab:confusion_topic} {\bf Topic level confusion matrices} on evaluation data (manifesto texts), classifiers were trained on speeches of 17th Bundestag.}
\vspace{0.5em}
\footnotesize
\begin{tabular}{lcccccc}
&& \multicolumn{5}{c}{\bf Predicted}\\
&& cducsu & fdp& gruene& linke& spd\\
\hline
\hline
\multirow{5}{*}{\rotatebox{90}{\pbox{2cm}{\centering {\bf True}}}} &cducsu &7& 0& 0& 0& 0\\
&fdp&0& 7& 0& 0& 0\\
&gruene&0& 0& 6& 0& 1\\
&linke&0& 0& 0& 7& 0\\
&spd&4& 0& 0& 0& 4\\
\end{tabular}
\end{table}


\begin{table*}[t]
\caption{
\label{tab:results_binary_17}
Classification performance on the binary prediction problem in the 17th legislative period, categorizing speeches into government (FDP/CDU/CSU) and opposition (Linke, Gr\"une, SPD).
}
\begin{center}
\begin{tabular}{lcccccccc}
& \multicolumn{4}{c}{\bf Held-out parliament speeches} & \multicolumn{4}{c}{\bf Party Manifestos}\\
    &         precision    &recall &  f1-score  & N    &         precision    &recall &  f1-score  & N\\
\hline \hline
government    &   0.83   &   0.84&      0.84     & 1037 & 0.49  &    0.59 &     0.54  &    4349\\
 opposition     &  0.86  &    0.86   &   0.86    &  1242 & 0.74 &     0.66  &    0.70   &   7726\\
\hline
avg / total   &    0.85 &     0.85 &     0.85  &    2279 & 0.65  &     0.63  &     0.64  &    12075\\

%
\end{tabular}
\end{center}
\end{table*}
%%
%%\begin{table*}[t]
%%\caption{
%%\label{tab:results_binary_18}
%%Classification performance on the binary prediction problem in the 18th legislative period, categorizing speeches into government (SDP/CDU/CSU) and opposition (Linke, Gr\"une).
%%}
%%\begin{center}
%%\begin{tabular}{lcccccccc}
%%& \multicolumn{4}{c}{\bf Held-out parliament speeches} & \multicolumn{4}{c}{\bf Party Manifestos}\\
%%    &         precision    &recall &  f1-score  & N    &         precision    &recall &  f1-score  & N\\
%%\hline \hline
%%government   &    0.88  &    0.95    &  0.92   &    786   &0.52   &   0.66 &     0.58  &    5972\\
%% opposition    &   0.86    &  0.71   &   0.78   &    346 & 0.69  &    0.56    &  0.62   &   8229\\
%%\hline
%%avg / total    &   0.88    &  0.88     & 0.87  &    1132 &  0.62   &   0.60    &  0.60 &    14201\\
%%%
%%\end{tabular}
%%\end{center}
%%\end{table*}

%
\subsection{Correlations between words and parties}\label{sec:word_party_correlations}
To get a more precise idea of what word features are discriminative in terms of the party affiliation label correlations between words and party affilation labels are computed; these are the words that are preferentially used or avoided by each respective party. Some unspecific stopwords are excluded. We list for each party some of the top positively correlated words, sorted in descendent order.
\paragraph{\bf Left party (linke)} Often used words include referrals to big companies ({\em konzerne}) and their profits ({\em profite}), the working class {\em beschaeftigte}, the social welfare program {\em hartz iv} as well as war ({\em krieg}).
\paragraph{\bf Green party (gruene)} Use words related to environmental damage ({\em klimaschaedlichen}), exploited low wage employees ({\em leiharbeitskraefte}) and pensions ({\em garantierente}).
\paragraph{\bf Social Democratic Party (SPD)} Uses mostly unspecific words related to the parliament and governmental processes ({\em Staatssekretaerin, Kanzlerin, Bundestagsfraktion}) and some words related to cutting of expenses ({\em Kuerzungen}).
\paragraph{\bf Christian Democratic Union/Christian Social Union (CDU/CSU)}
Often used words relate to a pro-economy attitude, such as competitiveness or (economic) development ({\em Wettbewerbsf\"ahigkeit, Entwicklung}) and words related to security ({\em Sicherheit, Stabilitaet}). 

\subsection{Predicting government status}\label{sec:sentiment_result}
Next to the party affiliation labels also government membership labels were used to train models that predict whether or not a text is from a party that belonged to a governing coalition of the Bundestag. In \autoref{tab:results_binary_17} the results are shown for the 17th legislative period. While the in-domain evaluation precision and recall values reach values close to 0.9, the out-of-domain evaluation drops again to values between 0.6 and 0.7. This is in line with the results on binary classification of political bias in the Canadian parliament \cite{Yu2008}. The authors report classification accuracies between 0.80 and 0.87, the accuracy in the 17th Bundestag was 0.85. A pronounced drop in performance was reported when applying the model to texts from a different domain (e.g. older texts or texts from another chamber). While we did not perform topic-level predictions in this binary setting, the party affiliation results in \autoref{tab:results_topic} indicates that a similar increase in out-of-domain prediction accuracy could be achieved when aggregating texts to longer segments. \\


\subsection{Sentiment correlates with political power} 
The pronounced drop in performance for the out-of-domain data suggests that there is a dependency between text features and labels in the training data that is not present in the out-of-domain data. One candidate feature for this dependency that can be easily examined is sentiment. Analysing the correlation between proxies of political power and text sentiment shows an interesting relationship between political power and sentiment. Political power was evaluated in terms of membership of the government as well as number of seats in the parliament. Correlating these indicators of political power with the mean sentiment of a party shows a strong positive correlation between speech sentiment and political power. This pattern is evident from the data in 
%\autoref{fig:party_sentiments} and 
in \autoref{tab:sentiments}: Government membership correlates with positive sentiment with a correlation coefficient of 0.98 and the number of seats correlates with 0.89.

\begin{table}[t]
\caption{
\label{tab:sentiments}
Correlation coefficient between average sentiment of political speeches of a party in the german Bundestag with two indicators of political power, a) membership in the government and b) the number of seats a party occupies in the parliament.
}
\begin{center}
\begin{tabular}{lcc}
   Sentiment vs. &          Gov. Member    &  Seats\\
\hline\hline
17th Bundestag    &  0.84 & 0.70\\
18th Bundestag   &  0.98 & 0.89\\
%
\end{tabular}
\end{center}
\end{table}




%\subsection{An example web application}
%To show an example use case of the above models a web application was implemented that downloads regularly all articles from some major German news paper websites\footnote{\url{http://www.spiegel.de/politik}, \url{http://www.faz.net/aktuell/politik}, \url{http://www.welt.de/politik}, \url{http://www.sueddeutsche.de/politik}, \url{http://www.zeit.de/politik}} and applies some simple topic modelling to them. For each news article topic, headlines of articles are plotted along with the predictions of the policy issue of an article and two labels derived deterministically from the 56 class output, a left right index and the political domain of a text, see \cite{leftright}. Within each topic it is then possible to get an ordered (from left to right) overview of the articles on that topic. A preliminary demo is live at \cite{fipidemo} and the code is available on github\cite{fipi}.
%\begin{figure}
%\begin{center}
%\includegraphics[width=10cm]{images/fipi-screenshot}
%%
%\end{center}
%\caption{
%\label{fig:fipi}
%A screen shot of an example web application using the political view prediction combined with topic modelling to provide a heterogeneous overview of a topic. }
%\end{figure}


\section{Conclusions, Limitations and Outlook}\label{sec:conclusion}
This study presents a simple approach for automated political bias prediction. The results of these experiments show that automated political bias prediction is possible with above chance accuracy in some cases. It is worth noting that even if the accuracies are not perfect, they are above chance and comparable with results of comparable studies \cite{Yu2008, Hirst2014}. While these results do not allow for usage in production systems for classification, it is well possible to use such a system as assistive technology for human annotators in an active learning setting.

One of the main limiting factors of an automated political bias prediction system is the availability of training data. Most training data sets that are publicly available have an inherent bias as they are sampled from a different domain. This study tried to quantify the impact of this effect.
For the cases in which evaluation data from two domains was available there was a pronounced drop in prediction accuracy between the in domain evaluation set and the out of domain evaluation set. This effect was reported previously for similar data, see e.g. \cite{Yu2008}. Also the finding that shorter texts are more difficult to classify than longer texts is in line with previous studies \cite{Hirst2014}. When considering texts of sufficient length (for instance by aggregating all texts of a given political topic) classification performance improved and in some cases reliable predictions could be obtained even beyond the training text domain.

Some aspects of these analyses could be interesting for social science researchers; three of these are highlighted here.
First the misclassifications of a model can be related to the changes in policy of a party. Such analyses could be helpful to quantitatively investigate a change in policy. Second analysing the word-party correlations shows that some discriminative words can be related to the political views of a party; this allows for validation of the models by human experts. Third when correlating the sentiment of a speech with measures of political power there is a strong positive correlation between political power and positive sentiment. While such an insight in itself might seem not very surprising this quantifiable link between power and sentiment could be useful nonetheless: Sentiment analysis is a rather domain independent measure, it can be easily automated and scaled up to massive amounts of text data. Combining sentiment features with other measures of political bias could potentially help to alleviate some of the domain-adaptation problems encountered when applying models trained on parliament data to data from other domains. \\


All data sets used in this study were publicly available, all code for experiments and the live web application can be found online \cite{fipi, fipidemo}.

\subsection*{Acknowledgements}
We would like to thank Friedrich Lindenberg for factoring out the \url{https://github.com/bundestag/plpr-scraper} from his Bundestag project. Michael Gaebler provided helpful feedback on an earlier version of the manuscript. \\
%

\newpage
\small{
\bibliographystyle{plain}
\bibliography{political_bias_prediction}
}


\end{document}
